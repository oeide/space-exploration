\documentclass[12pt]{book}
\usepackage[british]{babel}
\usepackage[utf8]{inputenc}
\usepackage[T1]{fontenc}
\usepackage[none]{hyphenat}
\usepackage{graphicx}
\usepackage{gensymb}
\usepackage{natbib}
\usepackage{enumitem}
\usepackage{fancyvrb}
\usepackage{color}
\usepackage{makeidx}
\usepackage{setspace}
\usepackage{authorindex}
\usepackage[stable]{footmisc}
\usepackage{minibox}
\usepackage{amsmath}
\usepackage{fmtcount}
\usepackage{soul}
\usepackage{listings}

\setlength{\textwidth}{12cm}
\setlength{\oddsidemargin}{2cm} 
\setlength{\evensidemargin}{2cm}
\setlength{\textheight}{21cm}
\setcounter{secnumdepth}{0}

\hyphenation{meta-phor-ised}
    
\begin{document}

\title{Unknown space project. Lab notes}

\author{Øyvind Eide}

\date{\today}

%\maketitle

%\tableofcontents

\setlength{\parindent}{0cm}
\setlength{\parskip}{3mm}
\let\stdsection\section
%\renewcommand\section{\newpage\stdsection}  

\chapter{Beyond Zeiger}

Practical work: \~/Documents/GitHub/space-exploration

\section{2024-01-21}

\begin{itemize}
\item Creating test 1: Double Between

\item Finding places through Norgeskart

\item Wenn Schnittfläcke ist leer: error

\item Modfified the width of between objects in pizzacut.py::Between

\end{itemize}

\subsection{Idea}

\begin{enumerate}
\item Find a movement passage in Schnitler
\item Enter all relational information, implicit and explicit
\item Add geolocation of all known places
\item If all places are known:
\begin{enumerate}
\item Run as many experiments as there are places
\item In each, one place is made unknown
\item Look a consequences of systematic choices as to if the place falls within the area predicted
\item What is the smallest possible area of prediction?
\end{enumerate}

\end{enumerate}

https://github.com/jjimenezshaw/Leaflet.UTM

\section{2024-01-24}

Can also use R for conversion: https://rdrr.io/cran/oce/

\subsection{Research plan}

\begin{enumerate}
\item Create a place record for each place name in SchnitlerTable.html.
\item Establish coordinates for each place using a methodology.\footnote{In this case, modern map studies based on the gazetteer in Schnitler}
\item Set parameter values for distance, span and direction based on distance in mile and contextual direction information.
\item For each place:
\begin{enumerate}
\item Generate rooms of possibilities from the place before and after.
\item See if the place falls within based on the parameter values used. 
\end{enumerate}
\end{enumerate}

\section{2024-01-25}

Introduce an outer iterative layer directly in the python scripts?

\section{2024-01-26}

Write a python script implementing the structure of the research plan

\section{2024-02-01}

Zeiger is using mathematical coordinates, not geographical.

\section{2024-03-25}

Modified main.py, pizzacut.py and draw.py so that a map is drawn also when there is no overlap. 

Steps:

\begin{lstlisting}
Space> python scripts/reearchPlan.py
Please input filepath (../file.csv): Documents/testSchnitler.csv
UnknownPlaces-master> python main.py
\end{lstlisting}

There is something weird with the distances.

And: the road does not in reality go eastwards.

\section{2024-09-20}

Start with distance only, with three values for Miile: 6, 8, 11,3

\section{2024-09-20}

It is now working with one pair of places, length of miile and with of span set by variables that can be changed. 

Rutten Field to bræcke gaard is too far away with my locations, well over 11,3 km.

Todo 1: make it work for all pairs. OK

Warning!!!!

Modified /Users/oeide/Library/Python/3.10/lib/python/site-packages/folium/folium.py





\bibliographystyle{chicago}
\addcontentsline{toc}{section}{References}
\bibliography{/Users/oeide_loc/Documents/skriving/references/refsOE}
\end{document}
